\documentclass[twoside,openright]{uva-bachelor-thesis}
\usepackage[utf8]{inputenc}
\usepackage[T1]{fontenc}
\usepackage[british]{babel}

%\usepackage[dutch]{babel}  % uncomment if you write in dutch
\usepackage{graphicx}
\usepackage{url}
\usepackage[hidelinks]{hyperref}
\usepackage{biblatex}
\addbibresource{thesis.bib}
\usepackage{lmodern}
%Enable for final compilation
%\usepackage[stretch=10]{microtype}



% Title Page
\title{Late Transmuxing:\\Improving caching in videostreaming}
\author{Jelte Fennema}
\supervisors{Dirk Griffioen (Unified Streaming), Robert Belleman (UvA)}
\signedby{Robert Belleman (UvA)}


\begin{document}
\maketitle

\begin{abstract}
\end{abstract}


\tableofcontents

\chapter{Introduction}
\section{Traditional setup}
\section{Proposed setup}
\subsection{Theoretical improvements over the traditional setup}

\chapter{Background}
\section{Different streaming protocols}
\section{HTTP Range Requests}
Range requests have been added to HTTP to serve parts of a file and not full
files.~\autocite{rangerequests}
\section{Caching}
\subsection{Caching range requests}
To cache range requests of any size, ranger \autocite{ranger} can be used.



\chapter{Implementation}
\section{The storage server}
\section{The caching server} \section{Variations}

\chapter{Experiments}
\section{Setup}
\section{Results}

\chapter{Discussion}
\section{Future Work}
\chapter{Conclusions}


\printbibliography{}


\end{document}
