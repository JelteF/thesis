\documentclass[twoside,openright]{uva-bachelor-thesis}
\usepackage[utf8]{inputenc}
\usepackage[T1]{fontenc}
\usepackage[british]{babel}

%\usepackage[dutch]{babel}  % uncomment if you write in dutch
\usepackage{graphicx}
\usepackage{url}
\usepackage[hidelinks]{hyperref}
\usepackage[style=ieee]{biblatex}
\addbibresource{thesis.bib}
\usepackage{lmodern}
%Enable for final compilation
%\usepackage[stretch=10]{microtype}
\usepackage{graphicx}
\usepackage{varioref}



% Title Page
\title{Late Transmuxing:\\Improving caching \\in video streaming}
\author{Jelte Fennema}
\supervisors{Dirk Griffioen (Unified Streaming), Robert Belleman (UvA)}
\signedby{Robert Belleman (UvA)}


\begin{document}
\maketitle

\begin{abstract}
\end{abstract}


\tableofcontents

\chapter{Introduction}
% Start with the grand scheme of things with respect to VOD
Streaming video reliably is no easy job. One significant problem is that there
are different video streaming protocols and not all of them are supported
equally by all players. This means a good video host should have its videos in
a lot of different formats to make sure they can be viewed in every browser. The
video streaming protocols have one thing in common, they all stream little video
sections instead of the whole file, because that can be done easiliy with
HTTP\autocite{http}. All this means that at some point the original video file
should be cut in smaller sections and those should be converted to the formats
that are needed.

\section{Unified Streaming}
% Probably not worth a section
This thesis is written in collaboration with Unified Streaming Platform, which
will be called USP in the future. Unified Streaming is a company that provides an easy to use
solution for streaming video. They have modules for a set of popular
webserver applications, Apache, Nginx, IIS and Lighttpd.


\section{Different streaming protocols}
% More focus on the protocols, less on the USP
One of the things the USP software tries to make easy is supporting the
different streaming protocols that are used by the different types of players
that exist. All these protocols require that the video is sent in a (slightly)
different format.

At the moment streaming in four different formats is supported by the USP
software. IIS Smooth Streaming (ISS) \autocite{smooth} is developed by
Microsoft and is used in Silverlight based players. HTTP Dynamic Streaming (HDS)
\autocite{hds} is developed by Adobe and is used in Flash based players. HTTP Live
Streaming (HLS) \autocite{hls} is developed by Apple and used on the iPhone, iPad
and Apple TVs. The last format is Dynamic Adaptive Streaming over HTTP (DASH),
which is developed as an ISO standard and is used in HTML5 video players.


\section{Traditional setups}
% Maybe less details
The most obvious setup is ofcourse generating these files once and then setting
up a file server that serves all these little files. This works but takes up
quite a bit of space since the same video is essentially multiple
times. This issue can be solved by storing the raw video and converting it on
the fly to the format needed when a request comes in.

However, in large streaming setups this is not enough for two reasons. The first
one is that the storage server might be far away from the viewer, which means
more time is spent waiting before watching a video. The second one is that the
storage server can be easily overwhelmed with requests, because its just one
server. Using a Content Distribution
Networkwith reverse proxy caching nodes is a simple solution to these problems
\autocites{cdns}{revproxy}.  Those nodes forward the request to the storage
server and cache the response. If another request comes for the same content a
node will just respond with the response in its cache. This works quite good,
since a small percentage of the videos account for a large percentage of the
views.
% For live streaming this also works fantastic since it's not even a whole video
% that is downloaded a lot, it is only the last part of that video.

The big problem of this setup is that, just like with the very first setup, it
stores the converted files instead of the raw files, only now in the cache
instead of the storage server, which has two downsides in this case. Again, the
server will contain the same content multiple times, only in this case the cache
server. The other downside, specific to this setup, is that the cache will also
request basically the same content from the storage. Those requests mean more
internal traffic and more waiting time for the viewer.

\section{Approach}
% Very short exlaination of the idea to solve the problem.






\chapter{Problem analysis}
% Explain in the normal setups in detail together with their
% shortcommings.
\section{Different configurations}
The servers contain a couple of different webserver application configurations
to do this segment generation at different stages. A common characteristic in
all these configurations is that the client connects with the proxy server and
that server will then request the content it needs to fulfill this request from
the storage server.


\subsection{The CDN configuration}
A configuration that is currently heavily in use is the CDN configuration. In
this configuration the proxy server will receive the request for the segment
from the client and passes that exact reqest along to the storage server. The
storage server will then generate the segment from the mp4 and send the segment
back to the proxy server.  The proxy server will then send the response back to
the client, but it will also cache the response. That way, when a client
requests the same segment it will be able to serve it from its cache directly,
instead of requesting it from the storage server.
See figure \vref{fig:cdn}, for a schematic overview of this configuration.
\begin{figure}
    \includegraphics{cdn_diagram.pdf}
    \caption{The CDN configuration}
    \label{fig:cdn}
\end{figure}



\subsection{The IsmProxyPass configuration}
Another configuration that is currently being used is the IsmProxyPass
configuration. This configuration does one thing quite different compared to the
CDN configuration. When the proxy server receives a request for a segment,
instead of passing it to the storage directly, it will send a series of range
requests to the storage server. Upon receiving one of those range requests, the
storage server will return the bytes from the mp4 file that are in the request
range. When the proxy server receives those bytes it will then use them to
generate the segment requested by the client. After this is done the segment
will be returned to the client. Another important difference between this
configuration and the CDN configuration is that there is no caching involved.
See \vref{fig:ismproxy}, for a schematic overview of this configuration.

\begin{figure}

    \includegraphics{ismproxy_diagram.pdf}
    \caption{The IsmProxyPass configuration}
    \label{fig:ismproxy}

\end{figure}







\chapter{Design considerations}
% Explain the search for technologies used instead of stating the decision and
% then explaining why.
To build the setup proposed in the introduction, existing technology can used
heavily. This chapter will describe those technologies and explain how they can
be utilized.


\section{Webserver application on the proxy server}
Nginx is chosen as the webserver application for the reverse proxy in the
proposed setup. Apache could also be used, however Nginx has some features that
probably make it a better choice for the proposed setup.

\subsection{Caching} \label{sec:caching}
Since the proposed setup depends heavily on caching in the reverse proxy, it is
important that the server application does this wel. This is the case for Nginx.
Apache also does a good job at it, but one key difference is that cache locking is
supported better in Nginx. Cache locking means that the same content will not be
requested multiple times from the upstream server when multiple requests ask for
the same content at the same time and there is no cache entry for that
request. Instead only one request will be proxied through to the storage server
and the others wait until the response for that one is cached. This can
significantly reduce the load on the storage server.

\subsection{Scripting}
% In server lua vs CGI
Another reason for choosing Nginx are its extensive scripting capabilities. It
is possible to enable Lua scripting in the server itself by installing a special
module\autocite{nginxlua}. This can be very usefull if advanced logic is needed
to create the proposed setup.


\section{Caching range requests}
% Maybe use subsections so it will be less of a wall of text.
Since IsmProxyPass uses range requests it is also important that those can be
cached. This is not directly supported in any webserver. The reason for
this is that it is not exactly trivial to do. Most caches work by supplying it
some data to cache, together with a key by which the data can be retrieved later
on.
Ideally one would cache the content using the key of the full file and add some
info about what range is actually cached. Then when another request for a range
comes in it would check in the cache if some of the requested bytes are in
there. If not all of them are there it should request just the bytes that are
not cached from the upstream server and then add those to the same cache for
future use. This hasn't been implemented anywhere yet. However, our proposed
setup is not the first that would benefit from caching range requests. So some
suboptimal solutions have been proposed for different use cases.

The most simple one, which is employed by Nginx by default,
is removing the range request header from the request when sending it to
upstream. This will request the whole file instead of just the requesting byte
range. Nginx will then cache this and return the whole file everytime a range is
requested. According to the RFC this behaviour is allowed, but this it is
clearly not behaviour that one would want, since the whole point of range
requests is that requesting the full file is not necessary.

A simple solution proposed on the Nginx forums\autocite{nginxcacheforum} is to
force Nginx to request the range from upstream and then adding the requested
range to the cache key. This basically means that a range request is handled by
the cache as if it were a seperate URL. The main downside of this solution is that
when overlapping byte ranges are requested they will both be cached in full.
This means this  solution works fine when the ranges that get requested are
always the same. However if the requested ranges are different almost every
time this solution is not acceptable since a lot of the cache wil contain the
same bytes, because of the overlap in the byte ranges.

% Check wat proxy_force_ranges doet en of het te gebruiken is in mijn versie van

Ranger\autocite{ranger} is another solution to the problem that tries to solve
the overlap problem. Everytime a range request comes in, instead of sending it
directly to the upstream server, it generates other range requests that together
contain all the bytes requested. These new range requests are always from the
same block size and they also start at the same byte position. These requests
will then be sent to upstream and after a response comes back they will be
cached as well.  This way, even though the initial range request can be any
range and can overlap with others, the requests sent to upstream will always
match previously sent requests, which makes sure that cache hits occur every
time part of a range is requested that was requested before even though it's not
exactly the same. There are only two small issues with this approach. The first
one is that more data is requested from the upstream server than is actually
needed to fulfill the request. However, this is probably not too severe when the
block size is chosen small enough. The second issue is that when large ranges
are requested, a lot of requests need to be send to the upstream server, because
the large request gets split into a lot of smaller ones. Which could mean some
unnecessary overhead.





\chapter{Implementation}
The created server setup consists of two servers, in this case virtual machines.
One that has the video files on disk, this will be called the storage server,
and one that serves as a reverse proxy, this will be called the proxy server.
The proxy server can also be called the cache server in the cases where it
caches proxied requests.

The client requests a specific segment of the video in a specific format, say
ISS. The storage server only contains the mp4 file of the video. At some point
in the server chain the segment needs to be generated from (part of) that mp4.



\section{The Late Transmuxing configuration}
The last configuration is the one this thesis is about. The late transmuxing
configuration. This configuration is a mix of the previous two configurations.
Stating it very simply, it is a version of the IsmProxyPass configuration with
caching built in. This is accomplished by running two different server
applications on the proxy server. A request of the client is initially handled
on the proxy server by Nginx. Nginx then passes the request along to Apache,
running on the same server. Apache then uses IsmProxyPass from the Unified
Streaming module to send the range requests mentioned in the previous
configuration.  However, instead of sending those requests to the storage server
directly, it sends them back to Nginx. Nginx then passes those requests along to
the storage server. The storage server will again respond with the byte ranges
requested.

When Nginx receives a response to one of its range requests it should to
cache that response. This can be done quite easily by using the second method
described in section \ref{sec:caching}, which was the method that meant changing
the cache key. The reason this method is sufficient and the third method is not
needed is because that method solves a problem that our use case doesn't have.
It solves the problem of overlapping byte ranges. For video streaming however,
this problem doesn't exist because the because the fragments that cen be
requested are fixed. This means the same ranges will always be requested and
overlaps in those ranges should be minimal.

Apart from caching the response Nginx will also return it to Apache. Apache will
then generate the originally requested segment and return that back to Nginx.
Nginx will then cache the segment and return it to the client. See figure
\vref{fig:latetrans} for a schematic overview of this setup.

\begin{figure}
    \includegraphics{latetrans_diagram.pdf}
    \caption{The Late Transmuxing configuration}
    \label{fig:latetrans}
\end{figure}

\subsection{Using two webserver applications on the same server}
% Explain deadlocks


\section{Configurations with less caching}
Apart from these three main configurations, there are also configurations
with caching disabled at certain points in the server chains. These
configurations are created to find out what effect of each cache is on the
different performance metrics.






\chapter{Experiments}
\section{Setup}
\subsection{Comcast}
\subsection{wrk}
\section{Results}

\chapter{Discussion}
\section{Future Work}
\chapter{Conclusions}


\printbibliography{}


\end{document}
