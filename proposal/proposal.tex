\documentclass{article}
\usepackage{graphicx}

\usepackage{tikz}
\usetikzlibrary{arrows}
\usepackage{hyperref}

\usepackage[utf8]{inputenc}
\usepackage{etoolbox}
\usepackage{amsmath}
\usepackage{graphicx}
\usepackage{minted}


\title{Thesis proposal\\Late transmuxing in caching environments: a better
experience through lower latency}
\author{Jelte Fennema\\10183159}

\date{\today}

\begin{document}
\maketitle

\section{Context}
Streaming video reliably isn't an easy job. A significant problem is that video
formats are not supported equally by all browsers. This means a good
video host should have its videos in a lot of different formats to make sure
they can be viewed in every browser.

In large streaming setups, caching is central to the user experience. A simple
content distribution network (CDN) might just employ reverse proxy caching nodes
and fetch all content from the origin layer. This works quite good, since a
small percentage of the videos account for a large percentage of the views. For
live streaming this also works fantastic since it's not even a whole video that
is downloaded a lot, it is only the last part of that video.

However, like in most cases this simple setup has some downsides. One of the
downsides of this setup stems from the problem of supporting multiple formats.
In this setup a popular video needs to be cached in every supported format on
the caching nodes. Because of this, the internal traffic used for to request a
video for caching is multiplied. Also the space required for caching a video is
multiplied. What worsens matters even more is that edges and origins may be
geographically separated thus introducing latency. When an edge has to wait for
the origin the viewer also waits.

\section{Research question}
A possible improvement to this setup is the use of ``Late transmuxing''. This
leverages the edge’s resources (cpu, ram) to multiplex all formats from a single
mezzanine format. This way wait times can be reduced to a single fetch and cache
hit ratios go up (as content can generated when it is not there). The second
assumption is lower internal bandwidth and better resource utilization.

It is unknown how much better this setup will perform. Although unlikely, it
might even perform worse than the traditional setup if the format conversion
takes longer than the network latency. The is why this project has two goals.
First a late transmuxing setup should be built. After this is done, the setup
should be tested and compared against a normal setup. Based on the testing the
setup could possibly be improved.

\section{Planning}

\begin{itemize}
    \item Week 14: Get the normal setup working
    \item Week 15-19: Change the normal setup to a late transmuxing setup
    \item Week 20-22: Test and compare performance of the setup
    \item Week 23-25: Write thesis
\end{itemize}


\end{document}
