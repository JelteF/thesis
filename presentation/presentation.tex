\documentclass[xcolor=svgnames]{beamer}

\usepackage[utf8]{inputenc}
\usepackage[T1]{fontenc}
\usepackage[english]{babel}

\usepackage{amsmath,amsfonts,graphicx}
\usepackage{beamerleanprogress}
\usepackage{tikz}
\usepackage{xparse}
\usetikzlibrary{calc}
\usetikzlibrary{graphdrawing,graphs,arrows.meta,shapes.misc,chains,positioning,shapes,quotes,automata,bending,fit,overlay-beamer-styles}
\usepackage{tkz-graph}

\hyphenation{ism-proxy-pass}

\newcommand{\reqseg}{"Request for segment"}
\newcommand{\reqrange}{"Requests for byte ranges needed to generate segment"}
\newcommand{\reqrangeone}{Requests for byte ranges}
\newcommand{\reqrangetwo}{needed to generate segment}

\newcommand{\clientnode}{
    \node[block] (cl) {Client};
    \node[right=3cm of cl] (cl_desc) {Wants to play segment with a video player};
    \graph [use existing nodes] {
        cl --[dotted] cl_desc;
    };
}
\newcommand{\labelpos}[1]{($(#1.north east) - (0.2, 0)$) }
\newcommand{\rectlabel}[3]{\node[#3] at \labelpos{#1} {#2}}
\newcommand{\container}[6]{\draw[#6] ($(#1.north west)+(#4)$) rectangle ($(#2.south east)+(#5)$) node[fitting node] (#3) {}}
\DeclareDocumentCommand{\contwithlabelmore}{m m m m m m m O{container} O{label}}{
    \onslide<#7->{
        \container{#1}{#2}{#3}{#5}{#6}{#8};
        \rectlabel{#3}{#4}{#9};
    }
}
\newcommand{\contwithlabel}[6]{
    \onslide<#6->{
        \container{#1}{#1}{#2}{#4}{#5}{container};
        \rectlabel{#2}{#3}{label};
    }
}

\newcommand{\nodewithdesc}[7]{
    \onslide<#6->{
        \node[block, #2] (#1) {#4};
    }

    \onslide<#7->{
        \node[#3 #1] (#1_desc) {#5};
        \graph [use existing nodes] {
            #1 --[dotted] #1_desc;
        };
    }
}

\newcommand{\containercolor}{red}
\newcommand{\nodecolor}{blue}
\makeatletter
\tikzset{
  fitting node/.style={
    inner sep=0pt,
    fill=none,
    draw=none,
    reset transform,
    fit={(\pgf@pathminx,\pgf@pathminy) (\pgf@pathmaxx,\pgf@pathmaxy)}
  },
  reset transform/.code={\pgftransformreset}
}
\makeatother

\tikzset{
    block/.style={
        rectangle,
        draw=\nodecolor,
        thick,
        text width=5em,
        align=center,
        rounded corners,
        minimum height=2em
    },
    container/.style={
        draw=\containercolor,
        thick,
        rounded corners,
    },
    label/.style={
        rectangle,
        draw=\containercolor,
        fill=white,
        rounded corners,
        anchor=east,
    },
}


\usepackage{tikzscale}

\usepackage{booktabs}
\usepackage{tabulary}
\usepackage{caption}
\setbeamerfont{caption}{size=\scriptsize}

\title[Late Transmuxing\hspace{2em}]{Late Transmuxing:\\Improving caching in video streaming}

\author[Jelte Fennema]{Jelte Fennema\\\footnotesize{Begeleider: Dirk Griffioen}}

\date{\today}

\institute{Universiteit van Amsterdam\and Unified Streaming}


\begin{document}

\maketitle

\section{Introductie}

\begin{frame}{Introductie}
  \begin{itemize}
      \item Een groot gedeelte van het internet
      \item Verschillende formaten en verschillende apparaten
      \item Opslag is duur
  \end{itemize}
\end{frame}

\begin{frame}{Unified Streaming}
  \begin{itemize}
      \item Unified Streaming is gespecialiseerd in video streaming software
      \item On the fly transmuxing
  \end{itemize}
\end{frame}

\section{Traditionele setups}

\begin{frame}{Single server setup}
    \begin{figure}

        \resizebox{\textwidth}{!}{%
            \begin{tikzpicture}[ ]
    \clientnode
    \pause

    \nodewithdesc{ap}{below=3cm of cl}{right=3cm of}{Apache}{Zet mp4 om naar segment}{3}{3};


    \contwithlabel{ap}{storage}{Opslag server}{-0.3,0.6}{2,-0.6}{2}

    \graph [use existing nodes] {
        cl ->[\reqseg] ap;
    };

    \onslide<1->

\end{tikzpicture}

        }
    \end{figure}
\end{frame}
\begin{frame}{Eigenschappen van de single server setup}
    \begin{itemize}
        \item Makkelijk op te zetten

        \item Veel latency

        \item Server wordt makkelijk overbelast

    \end{itemize}
\end{frame}

\begin{frame}{CDN setup}
    \begin{figure}

        \resizebox*{!}{\dimexpr\textheight-2\baselineskip\relax}{%
            \begin{tikzpicture}[ ]

    \clientnode

    \pause
    \onslide<3->{
        \node[block, below=3cm of cl] (ng) {Nginx};
    }


    \contwithlabel{ng}{cache}{Proxy server}{-0.3,0.6}{1.5,-0.6}{2}
    \onslide<5->{
        \node[block, below=3cm of ng] (ap) {Apache};
        \node[right=3cm of ap] (ap_desc) {Zet mp4 om naar segment};
    }
    \contwithlabel{ap}{storage}{Opslag server}{-0.3,0.6}{2,-0.6}{4}

    \onslide<6->{
        \node[right=3cm of ng] (ng_desc) {Cachet de ontvangen response};
    }
    \graph [use existing nodes] {
        cl ->[\reqseg] ng ->[\reqseg, visible on=<4->] ap;
        ng --[dotted, visible on=<6->] ng_desc;
        ap --[dotted, visible on=<5->] ap_desc;
    };

    \onslide<1->
\end{tikzpicture}

        }
    \end{figure}
\end{frame}

\begin{frame}{Eigenschappen van de CDN setup}
  \begin{itemize}
  \item Moeilijker op te zetten
  \item Geen server overbelasting
  \item Latency kan worden verminderd
  \item Slaat alle formaten op in cache
  \end{itemize}
\end{frame}

\begin{frame}{IsmProxyPass setup}
    \begin{figure}

        \resizebox*{!}{\dimexpr\textheight-2\baselineskip\relax}{%
            \begin{tikzpicture}[ ]

    \clientnode

    \pause
    \onslide<3->{
        \node[block, below=3cm of cl] (ng) {Nginx};
    }
    \contwithlabel{ng}{proxy}{Proxy server}{-0.3,0.6}{1.5,-0.6}{2}

    \onslide<5->{
        \node[block, below=3cm of ng] (ap) {Apache};
        \node[right=3cm of ap] (ap_desc) {Returns the requested bytes of the mp4};
    }

    \contwithlabel{ap}{storage}{Storage server}{-0.3,0.6}{2,-0.6}{4}

    \onslide<6->{
        \node[right=3cm of ng] (ng_desc) {Creates segment from range requests};
    }
    \graph [use existing nodes] {
        cl ->[\reqseg] ng ->[\reqrange, visible on=<4->] ap;
        ng --[dotted, visible on=<6->] ng_desc;
        ap --[dotted, visible on=<5->] ap_desc;
    };

    \onslide<1->

\end{tikzpicture}


        }
    \end{figure}
\end{frame}

\begin{frame}{Eigenschappen van de IsmProxyPass setup}
  \begin{itemize}
  \item Ook moeilijk op te zetten
  \item Geen server overbelasting
  \item Veel latency
  \item Domme opslag kan worden gebruikt (e.g. Amazon S3)
  \end{itemize}
\end{frame}


\section{Ontwikkelde setup}

\begin{frame}{Ontwikkelde setup}
    \begin{itemize}
        \item Een combinatie van de CDN en de IsmProxyPass setup
        \item Minder intern verkeer
        \item Efficiënter gebruik van cache
    \end{itemize}
\end{frame}

\begin{frame}{Late Transmuxing setup}
    \begin{figure}

        \resizebox*{!}{\dimexpr\textheight-2\baselineskip\relax}{%
            \begin{tikzpicture}[ ]

    \clientnode
    \pause

    \nodewithdesc{sp}{below=2cm of cl}{right=7cm of}{Segment proxy}{Cachet
    het uiteindelijke segment}{3}{10}

    \nodewithdesc{ismp}{below right=1cm and 3cm of sp}{right=1.768cm
    of}{IsmProxyPass}{Maakt het segment van range requests}{5}{9}
    \nodewithdesc{rp}{below left=1cm and 3cm of ismp}{right=7cm of}{Range request
    proxy}{Cachet byte range responses}{6}{8}

    \nodewithdesc{ap}{below=3cm of rp}{right=3cm of}{Apache}{Geeft de
    opgevraagde bytes van de mp4 terug}{7}{7};

    \contwithlabelmore{sp}{rp}{ng_cont}{Nginx}{-0.3,0.6}{0.5,-0.6}{3}
    \contwithlabel{ismp}{ap_cont}{Apache}{-0.3,0.6}{1,-0.6}{4}

    \contwithlabel{ng_cont}{cache}{Proxy server}{-0.3,0.6}{6,-0.6}{2}

    \contwithlabel{ap}{storage}{Opslag server}{-0.3,0.6}{2,-0.6}{6}

    \graph [use existing nodes] {
        cl ->[\reqseg,pos=0.2] sp ->[\reqseg, pos=0.2, visible on=<4->] ismp ->[align=left,
        "\reqrangeone \\\reqrangetwo",  pos=0.55, visible on=<5->] rp
        ->[\reqrange, pos=0.58, visible on=<6->] ap;
    };

    \onslide<1->
\end{tikzpicture}

        }
    \end{figure}

\end{frame}


\section{Experimenten}
% \begin{frame}{Experimenten}
%     \begin{itemize}
%         \item Virtual machines worden gebruikt voor de verschillende servers
%         \item Bandbreedte beperken met Comcast
%         \item wrk
%     \end{itemize}
% \end{frame}

\begin{frame}{Experimenten}
    \begin{itemize}
        \item Oude setups
            \begin{itemize}
                \item IsmProxyPass (IPP)
                \item CDN
            \end{itemize}
            \pause

        \item New setups
            \begin{itemize}
                \item Late transmuxing met één cache (LT-single)
                \item Late transmuxing met twee caches (LT-double)
            \end{itemize}
            \pause

        \item Controle setups
            \begin{itemize}
                \item CDN zonder caching (CDN-nocache)
                \item Late transmuxing zonder caching (LT-nocache)
            \end{itemize}

    \end{itemize}
\end{frame}

\begin{frame}{Performance metrics}
    \begin{itemize}
        \item Throughput, gemeten in MB/s en requests/s
        \item Latency
        \item Intern verkeer, gemeten in bytes en aantal requests
        \item Cache gebruik
    \end{itemize}
\end{frame}


\begin{frame}{Cache staat}
    \begin{itemize}
        \item Lege cache
        \item Cache gevuld door een ander formaat te downloaden
        \item Cache gevuld door hetzelfde formaat te downloaden\pause
        \item Één volledige download bij de eerste twee
        \item Zoveel mogelijk downloads bij de laatste
    \end{itemize}
\end{frame}

\section{Resultaten}
\newcommand{\frameplot}[3]{%
    \begin{frame}{Resultaten met een #1}
        \input{plots/#2_#3.tex}
    \end{frame}
}

\newcommand{\plotframes}[2]{%
    \frameplot{#1}{#2}{cache_usage}
    \frameplot{#1}{#2}{internal_requests}
    \frameplot{#1}{#2}{internal_mb}
    \frameplot{#1}{#2}{mbps}
    \frameplot{#1}{#2}{requests_per_second}
    \frameplot{#1}{#2}{latency_mean}
}

\plotframes{lege cache}{first_time}
\plotframes{cache gevuld met hetzelfde formaat}{second_time}
\plotframes{cache gevuld met een ander formaat}{after_other}

\begin{frame}{Samenvatting van de resultaten}
    \begin{itemize}
        \item Stuff
    \end{itemize}
\end{frame}


\section{Conclusie}

\begin{frame}{Conclusie}
  \begin{itemize}
      \item Een nuttige toevoeging aan de huidige setups
      \item Keuze tussen snelheid en cache gebruik
      \item Kan gebruikt worden met domme opslag
  \end{itemize}
\end{frame}


\begin{frame}{Vragen}
    Zijn er nog vragen?


\end{frame}

\end{document}
